\documentclass[]{report}
\usepackage{biblatex}
\usepackage[czech]{babel}
% Title Page
\title{Klient IMAP s podporou TLS}
\author{Michael Peštuka}


\begin{document}
\maketitle

\begin{abstract}
\end{abstract}

\section{Teorie }


\subsection{Šifrování SSL/TLS}
\subsection{Protokol IMAP4rev1}


¨
\section{Návrh a imlementace}
Program je rozdělený do několika souborů a tříd, které implementují potřebnou funkcionalitu. Všechny zdrojové soubory se nachází v adresáří ./src
\subsection{imapcl.cpp}
Soubor imapcl.cpp obsahuje funkci Main a slouží jako vstupní bod programu. Argumenty zadané uživatelem jsou zpracovány a uloženy v instanci třídy Argparser. Na základě argumentů je navázáno spojení, reprezentované instancí třídy TLSConnection nebo UnsecuredConnection. Následně je zahájena komunikace se serverem na bázi stavového stroje popsaného třídou FSM.
\subsection{argparser.cpp/.hpp}
Třída argparser zpracovává argumenty programu z příkazové řádky a jejich hodnoty ukládá v instanci této třídy. Zpracování argumentů je řešeno vlastním parserem.
\subsection{connection.cpp/.hpp}
Třída Connection slouží jako abstraktní třída ze které dědí třídy TLSConnection a UnsecuredConnection, které spravují připojení na server, čtení a zapisování dat na něj pro TLS/SSL a nezabezpečené spojení respektive.
\subsection{fsm.cpp/.hpp}
Třída ovládající běh programu a jeho komunikace se serverem. Běh programu je modelován Mooreovým konečným automatem, jehož vstupy jsou zprávy obdržené ze serveru a jeho výstupy zprávy odeslané serveru.
\subsection{fileops.cpp/.hpp}
Pomocné třídy na práci se soubory, konkrétně pro zápis obsahu e-mailů a čtení přihlašovacích údajů uživatele. 
\section{Funkcionalita programu}
\subsection{Výběr e-mailů ze serveru}
UID stahováných e-mailů jsou získávány příkazem "UID SEARCH ALL", v případě stahování pouze nových e-mailů "UID SEARCH NEW"
\subsection{Stahování e-mailů ze serveru}
E-maily jsou získány příkazem "UID FETCH BODY[]", hlavičkové soubory příkazem "UID FETCH BODY.PEEK[HEADER]", tím pádem se při stahování hlaviček nezmění stav SEEN e-mailu.
\subsection{Ukládání e-mailů ze serveru}
Zprávy ze serveru jsou ukládány do souborů pojmenovaných formátem "user@email.com\_mailbox\_UID\_UIDVALIDITY", při čtení pouze hlaviček je k názvu souboru přidáno "\_HEADER". Hlavičky stažené při použití parametru "-h" a celé e-maily jsou tím pádem ukládány zvlášť a nijak se nepřepisují.

\section{Použití programu}

Použití: imapcl server [-p port] [-T [-c certfile] [-C certaddr]] [-n] [-h] -a auth\_file [-b MAILBOX] -o out\_dir

Parametry v hranatých závorkách jsou volitelné.
Pořadí parametrů je libovolné. Popis parametrů:
\begin{itemize}
	 


\item První parametr je vždy název serveru (IP adresa, nebo doménové jméno).
\item -p port specifikuje číslo portu na serveru. Výchozí hodnota je 143, při použití TLS 993.
\item -T zapíná šifrování přes SSL/TLS
\item -c určuje soubor s certifikáty, který se použije pro ověření platnosti certifikátu SSL/TLS předloženého serverem.
\item -C určuje adresář, ve kterém se mají vyhledávat certifikáty, které se použijí pro ověření platnosti certifikátu SSL/TLS předloženého serverem. Výchozí hodnota je /etc/ssl/certs.
\item -n provádí stažení pouze nových zpráv
\item -h provádí stažení pouze hlaviček zpráv.
\item -a auth\_file odkazuje na soubor s přihlašovacími údaji uživatele
\item -b určuje název poštovní schránky, ze které budou staženy maily. Výchozí hodnota je INBOX.
\item -o out\_dir specifikuje adresář, do kterého budou stažené maily uloženy.
\end{itemize}

\section{Testování aplikace}
\printbibliography

\end{document}          
